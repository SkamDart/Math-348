\documentclass[12pt]{article}
\usepackage{amsmath,amsthm,amssymb}
\newcommand{\N}{\mathbb{N}}
\newcommand{\Z}{\mathbb{Z}}
\newcommand{\R}{\mathbb{R}}
 
\newenvironment{theorem}[2][Theorem]{\begin{trivlist}
\item[\hskip \labelsep {\bfseries #1}\hskip \labelsep {\bfseries #2.}]}{\end{trivlist}}
\newenvironment{lemma}[2][Lemma]{\begin{trivlist}
\item[\hskip \labelsep {\bfseries #1}\hskip \labelsep {\bfseries #2.}]}{\end{trivlist}}
\newenvironment{exercise}[2][Exercise]{\begin{trivlist}
\item[\hskip \labelsep {\bfseries #1}\hskip \labelsep {\bfseries #2.}]}{\end{trivlist}}
\newenvironment{problem}[2][Problem]{\begin{trivlist}
\item[\hskip \labelsep {\bfseries #1}\hskip \labelsep {\bfseries #2.}]}{\end{trivlist}}
\newenvironment{question}[2][Question]{\begin{trivlist}
\item[\hskip \labelsep {\bfseries #1}\hskip \labelsep {\bfseries #2.}]}{\end{trivlist}}
\def\imp{\rightarrow}
\newenvironment{level}%
{\addtolength{\itemindent}{2em}}%
{\addtolength{\itemindent}{-2em}}

 
\begin{document}
  
\title{Seminar On Bounded Sets}
\author{Cameron Dart\\ 
Math 348} 

\maketitle 

\begin{question}{4}
Find the lub and glb of the following sets.
\end{question}
\begin{exercise}{$A = \{x \vert x = 2^{-p} + 3^{-q}, \, \forall p,q \in \mathbb{N} \}$}
let $(p,q) = (1,1)\\ \therefore x = 2^1 + 3^2 = 5 $\\
$lub(A) = 5$\\ \\
let $p,q = b$\\
$x = \lim_{b \rightarrow \infty} 2^{-b}+3^{-b}$\\
$x = 0$\\
$\therefore$ glb(A) = 0
\end{exercise}
\begin{exercise}{$B=\{ x \vert (\, x \in (0,1)\,) \wedge (x \in \mathbb{R}) \}\\$}
$glb(B) = 0$\\
$lub(B) = 1$
\end{exercise}

\begin{question}{6}
Which of the following statements are true and which are false? Give adequate reasons for you answer.
\end{question}
\begin{exercise}B
$(\forall r \in \mathbb{R})(\exists B \subset Q)(r = glb(B))$\\
False.\\
Since $\mathbb{R}$ is uncountable by Cantor's Diagonal argument and $\mathbb{Q}$ is countable also by Cantor's Diagonal argument. It cannot be true that there is a map from all real numbers to a set in which the glb of that set is a real number.
\end{exercise}
\begin{exercise}D
If the greatest lower bound of a set of real numbers exists but is not a member of the set, then the set must be infinite, and have a subsequence that converges to its greatest lower bound.\\ \\
Let $A$ be an infinite set and $A_n$ be a subsequence\\
$(\alpha = glb(A)) \wedge (\alpha \not \in A) \implies (|A| = \infty)\wedge (\lim_{An \rightarrow \infty} = glb(A)) $\\
True.\\
By definition of glb, $(glb(A) = \alpha)(\forall x \in A)(\alpha \leq x)$ however, it is not necessarily true that $(\alpha \in A)$\\
If $(|A| = n)$ where $(n \in \mathbb{Z})$ and is finite then $\alpha \in A$\\
If this is not the case then there must exist some subsequence $A_n$ such that as
$A_n \rightarrow \infty = L $ where $L=\alpha$
\end{exercise}
\begin{question}{7}
Prove that the cubic equation $x^3-x-1=0$ has a real root by showing that any root of the equation is the lub of a suitable set.\\ \\
let $A = \{ x \vert x^3 - x - 1 < 0\}$, $c = glb(A)$, $f(x) = x^3-x-1$\\ \\
Since $f$ is continuous on the interval $(-\infty, \infty)$ and $(f(0) = -1) \wedge (f(2) = 3)$
by the Intermediate value theorem we know that f has is a real root somewhere on the interval of [0,2] \\ \\
Choose this 'suitable set' to be $S$ where $S = \{ x \vert x^3-x-1 < 0\}$ and $c$ is the lub of $S$.
We can use binary search on different values of $x$ where $x \in \mathbb{R}$ to determine the value of $C$.
%Next, I will prove by contradiction that $c^3 - c - 1 = 0$, or there exists a solution to the equation and that solution is the lub of the set.\\
%let $g(c) = c^3 - c - 1 \neq 0$\\
%The cubic function $g$ is already depressed so we can calculate this root by 
\end{question}
\end{document}