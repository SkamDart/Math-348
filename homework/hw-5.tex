\documentclass[12pt]{article}
\usepackage{amsmath,amsthm,amssymb}
\newcommand{\N}{\mathbb{N}}
\newcommand{\Z}{\mathbb{Z}}
 
\newenvironment{theorem}[2][Theorem]{\begin{trivlist}
\item[\hskip \labelsep {\bfseries #1}\hskip \labelsep {\bfseries #2.}]}{\end{trivlist}}
\newenvironment{lemma}[2][Lemma]{\begin{trivlist}
\item[\hskip \labelsep {\bfseries #1}\hskip \labelsep {\bfseries #2.}]}{\end{trivlist}}
\newenvironment{exercise}[2][Exercise]{\begin{trivlist}
\item[\hskip \labelsep {\bfseries #1}\hskip \labelsep {\bfseries #2.}]}{\end{trivlist}}
\newenvironment{problem}[2][Problem]{\begin{trivlist}
\item[\hskip \labelsep {\bfseries #1}\hskip \labelsep {\bfseries #2.}]}{\end{trivlist}}
\newenvironment{question}[2][Question]{\begin{trivlist}
\item[\hskip \labelsep {\bfseries #1}\hskip \labelsep {\bfseries #2.}]}{\end{trivlist}}
\def\imp{\rightarrow}
\newenvironment{level}%
{\addtolength{\itemindent}{2em}}%
{\addtolength{\itemindent}{-2em}}

 
\begin{document}
  
\title{Homework 5}
\author{Cameron Dart\\ 
Math 348} 
 
\maketitle

\begin{question}{6.8}
Compute gcd of (126,224) and (221,299).\\
\begin{align}
gcd(126,224) &= gcd(126,98)\\ 
&= gcd(28,98) \\
&= gcd(28,14) \\
&= 14 \\
\\ 
(126)(4) + (224)(-7) &= 14\\
896 - 882 &= 14\\
14 &= 14
\end{align}
\begin{align}
gcd(221,299) &= gcd(221,78)\\ 
&= gcd(65,78) \\
&= gcd(65,13) \\
&= 13 \\
\\ 
(221)(-4)+(299)(3)&= 13\\
-884+897 &=13\\
13 &= 13\\
\end{align}
\end{question}
%
\begin{question}{6.9}
Find all solutions for the diophantine equations.\\
%
\textbf{a)}$17x+13y=200$
\begin{enumerate}
	\item gcd$(17,13) = 1 $, $ 1 | 200 \therefore$ a solution exists.
	\item $17x+13y=1$
	\begin{level}
		\item let $x=-3,y=4$
		\item $17(-3)+13(4)=1$ 
		\item $1=1$
	\end{level}
	\item soln: (x,y) = $200*(-3,4)$
	\begin{level}
		\item soln: (x,y) = $(-600,800)$
	\end{level}
	\item let S be the set of solutions to the diophantine equation.
	\item $S = \{(-600+15k, 800+17k )$, $k \in \Z \}$
\end{enumerate}
%
\textbf{b)}$21x+15y=93$
\begin{enumerate}
	\item $\gcd(21,15) = 3\, , \, 3 | 93 \therefore $ a solution exists.
	\item $\frac{21x+15y=93}{3}$
	\item $7x+5y=31$
	\begin{level} 
		\item $7x+5y=1$
		\item let $x=3,y=-4$
		\item soln: (x,y)$=31*(3,-4)$
		\item soln: (x,y)$=(93,-124)$
	\end{level}
	\item let S be the set of solutions to the diophantine equation.
	\item $S = \{(93+5k, -124-7k) $, $k \in \Z \}$ 
\end{enumerate}

\textbf{c)}$60x+42y=104$
\begin{enumerate}
	\item $\gcd(60,42)=6$,$6 \not | 104 \therefore $ no integer solution exists.
\end{enumerate}

\textbf{d)}$588x+231y=63$\\
\begin{enumerate}
	\item $\gcd(588,231)=21$ , $21 | 63 \therefore$ a solution exists
	\item $\frac{588x+231y=63}{21}$
	\item $28x+11y=3$
	\item $28x+11y=1$
	\item let $x=2,y=-5$
	\item soln (x,y)=$3*(2,-5)$
	\item soln (x,y)=$(6,-15)$
	\item let S be set of solutions to the diophantine equation.
	\item $S = \{(6+11k, -15-28k)$, $k \in \Z \}$
\end{enumerate}
\end{question}
%%%
\begin{question}{6.17}Prove $\gcd(a+b,a-b)=\gcd(2a,a-b)=\gcd(a+b,2b)$\leavevmode
\begin{lemma}{A}
	Assume $p,q,r$ are all integers such that $p,q,r \neq 0$\\
	Let $p|q$ and $p|r$ evenly\\
	$p|q \, \wedge \, p|r \, \implies p|q+r$\\
	Since $p|q \implies (\exists k_1)(q=pk_1)$\\
	$p|r \implies (\exists k_2)(r=pk_2)$\\
	$\therefore q+r = p(k_1+k_2)$\\
	$q+r=p(k_3)$ for some integer $k_3=k_1+k_2$\\
	By definition of divisibility $p|q+r$\\
\end{lemma}
\begin{lemma}{B}
	Let $a,b,c$ be integers such that $a,b,c \neq 0$\\
	$a|b \implies a|bc$\\
	$b=ar$ where $r \in \mathbb{Z}$\\
	$bc=arc=(ar)c$\\
	Since $rc$ is an integer $a|bc$
\end{lemma}
\begin{proof}\leavevmode
\begin{enumerate}
	\item Let $d_1|(a+b)\,$ and $d_1|(a-b)$
	\begin{level}
		\item $(d_1 | (a+b)+(a-b))$
		\item $(d_1 | 2a )$ \hfill(By lemma A)
		\item $(d_1 | a-b )$ \hfill(By our original assumption)
		\item $(d_1 | b)$ \hfill(By lemma A)
		\item $(d_1 | 2b)$ \hfill(By lemma A and B)
	\end{level}
	\item Let $\gcd(2a,a-b)=d_2$
	\begin{level}
		\item $d_2 | 2a$
		\item $d_2 | a-b$
		\item Thus $d_1 = d_2$ since $d_1 , d_2$ both divide $2a \wedge a-b $
	\end{level}
	\item Let $d_3 | a+b,2b$
	\item Hence $d_1 = d_3$ since both $d_1 \wedge d_3$ divide $a+b \wedge 2b$ 
	\item Lastly, $d_1 = d_2 \wedge d_1 = d_3 \implies d_2 = d_3$
	\item Thus, $gcd(a+b,a-b)=\gcd(2a,a-b)=\gcd(a+b,2b)$
\end{enumerate}
\end{proof}
\end{question}
%%
\begin{question}{6.18} Suppose $\gcd(a,b)=1$.
\begin{proof}
$\gcd(a^2,b^2)=1$
	\begin{enumerate}
		\item $gcd(a,b)=1 \, \implies \, (\exists x,y \in \,\Z) \, (ax+by=1)$ \hfill(Integer Combination)
		\item $(ax+by)^3=1^3$ \hfill(Cube both sides)
		\item $a^3x^3+3a^2x^2by+3axb^2y^2+b^3y^3=1$ \hfill(Expand)
		\item $a^2(x^3+3x^2by)+b^2(3axy+by^3)$ \hfill(Factor)
		\item Let $m=(x^3+3x^2by),\, n=(3axy+by^3)$ \hfill(Declare Vars)
		\item $a^2m+b^2n=1$ \hfill(Substitute back in) 
		\item $gcd(a^2,b^2) = 1$ \hfill(Definition of GCD)
	\end{enumerate}
\end{proof}
%%
\begin{proof}
$\gcd(a,2b)\neq 1$\\
I will prove that $gcd(a,b)=1 \nRightarrow \, \gcd(a,2b)=1$ using contradiction. \\First, assume $(\forall a,b \in \Z \neq 0)(gcd(a,b)=1 \wedge gcd(a,2b)=1)$. In other words that both $(a,b) \wedge (a,2b)$ respectively are relatively prime. \\So $gcd(a,b)=1 \Rightarrow \, \gcd(a,2b)=1$ 
\begin{enumerate}
	\item $\gcd(a,b)=1$ \hfill(Given)
	\item Assume $gcd(a,2b)=1$ \hfill(Assumption)
	\item Let $a=2, b=5$ \hfill(Specify Individual Case)
	\item $\gcd(2,5) = 1 \, \therefore \, a,b$ are relatively prime. \hfill(Compute gcd)
	\item Now let's consider $\gcd(a,2b)=1$ \hfill(Assumption)
	\item $\gcd(2,2*5)=1$ \hfill(Multiplication)
	\item $\gcd(2,10)=2\neq 1$ \hfill(Compute gcd)
\end{enumerate}
This contradicts our original assumption $(\forall a,b \in \Z \neq 0)(gcd(a,b)=1 \wedge gcd(a,2b)=1)$\\
Hence it must be true that $\gcd(a,b) \nRightarrow gcd(a,2b)=1$\\
\end{proof}
\end{question}
%%
\begin{question}{6.28} Suppose that $\gcd(a,b)=1\, , \, a|n\, ,\, b|n$.
\end{question}\leavevmode
\begin{proof}$ab\,|\,n$
    \begin{enumerate}
    \item $a|n\, ,\, b|n$ \hfill(Given)
    \begin{level}
    	\item $(\exists m,n) (c=am,\, c=bn)$ \hfill(Definition of divides)
	\end{level}    
    \item $\gcd(a,b)=1 \therefore \, a,b$ are \textit{relatively prime} \hfill(Given)
    \begin{level}
    	\item $\exists s,t \in \Z$ such that,  $as+bt=1$ \hfill(Integer Combination of $a,b$)
    	\item $c(as+bt)=c$ \hfill(Multiply both sides by $c$)
    	\item $cas+cbt=c$ \hfill(Distributive Property of Multiplication)
    	\item $(bn)as+(am)bt=c$ \hfill(Substitutions from 2)
    	\item $ab(ns+mt)=c$ \hfill(Factor out $ab$)
    \end{level} 
    \item Let $u=ns \, , \, v = mt, \, z = (u+v)$ \hfill(Reassign Variables)
	\begin{level}
		\item $ab(u+v) = c$  \hfill(Rewrite $u=ns,v=mt$)
		\item $abz = c$ \hfill(Rewrite $z =(u+v)$)
		\item $ab|c$ \hfill(Definition of Divides)
 	\end{level}  
	\end{enumerate}
As shown through a direct proof, if $ \gcd(a,b)=1 \wedge \, a|n \,\land\, b|n \implies ab|n$
\end{proof}

\end{document}