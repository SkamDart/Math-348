\documentclass[11pt,a4paper]{book} 
\usepackage[utf8]{inputenc}
\usepackage[T1]{fontenc}
\usepackage{fourier}


\usepackage{amsmath}
\usepackage[thmmarks,  thref, amsmath]{ntheorem}

\theoremstyle{plain}
\newtheorem{thm}{Theorem}
\newenvironment{lemma}[2][Lemma]{\begin{trivlist}
\item[\hskip \labelsep {\bfseries #1}\hskip \labelsep {\bfseries #2.}]}{\end{trivlist}}
\theoremheaderfont{\itshape}
\theorembodyfont{\upshape}
\newtheorem{case}{Case}
\theoremstyle{nonumberplain}
\theoremheaderfont{\scshape}
\theorembodyfont{\upshape}
\theoremsymbol{\scshape Q. E. D.}
\theorempostwork{\setcounter{case}{0}}
\newtheorem{proof}{Proof}

\begin{document}

\begin{thm}
If $x,y,m \in \mathbb{Z}$ then $gcd(x,y) = gcd(x,y-mx)$
\end{thm}
\begin{lemma}{A}$p|q \, \wedge \, p|r \, \implies p|q+r$\\
	Assume $p,q,r$ are all integers such that $p,q,r \neq 0$\\
	Let $p|q$ and $p|r$ evenly\\
	Since $p|q \implies (\exists k_1)(q=pk_1)$\\
	$p|r \implies (\exists k_2)(r=pk_2)$\\
	$ q+r = p(k_1+k_2)$\\
	$q+r=p(k_3)$ for some integer $k_3=k_1+k_2$\\
	By definition of divisibility $p|q+r$\\
\end{lemma}
\begin{lemma}{B}
	Let $a,b,c$ be integers such that $a,b,c \neq 0$\\
	$a|b \implies a|bc$\\
	$b=ar$ where $r \in \mathbb{Z}$\\
	$bc=arc=(ar)c$\\
	Since $rc$ is an integer $a|bc$
\end{lemma}
\begin{proof}
There are two cases that must be considered for this proof.
\begin{case}
where $x=y=0$\\
Let $d_0 | x \wedge y$ 
\begin{align*}
gcd(0,0) &= gcd(0,0-0)\\
&= gcd(0,0)
\end{align*}
Hence, it must be true for this case.
\end{case}
%%%%%%%%%%%%%%
\begin{case}
Assume at least one of $x,y$ is non zero\\
Suppose $d_1|x$ and $d_1|y$\\
Let $n = -mx$\\
By Lemma B it is true that $d_1|n$\\
Since $d_1|x \wedge d_1|n$ by Lemma A it must be true that $d|y+n$\\
Hence $d_1 | y+n$\\
$d_1|y-mx$ 
\end{case}
\end{proof}
\end{document} 