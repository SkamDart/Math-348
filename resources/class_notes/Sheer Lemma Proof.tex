\documentclass[11pt,a4paper]{book} 
\usepackage[utf8]{inputenc}
\usepackage[T1]{fontenc}
\usepackage{fourier}


\usepackage{amsmath}
\usepackage[thmmarks,  thref, amsmath]{ntheorem}

\theoremstyle{plain}
\newtheorem{thm}{Theorem}

\theoremheaderfont{\itshape}
\theorembodyfont{\upshape}
\newtheorem{case}{Case}

\theoremstyle{nonumberplain}
\theoremheaderfont{\scshape}
\theorembodyfont{\upshape}
\theoremsymbol{\scshape Q. E. D.}
\theorempostwork{\setcounter{case}{0}}
\newtheorem{proof}{Proof}

\begin{document}

\begin{thm}
If $x,y,m \in \mathbb{Z}$ then $gcd(x,y) = gcd(x,y-mx)$
\end{thm}
\begin{proof}
Given $m \in \mathbb{Z}$\\ 
It must be true that\\
$(y-mx) = (y\mod x)$\\
Thus we are really trying to prove.. \\
\begin{align*}
gcd(x,y) &= gcd(x,y-mx)\\
&= gcd(x,y mod x)\\ 
\end{align*}

That being true, there are two cases that must be considered for this proof.
\begin{case}
where $x=y=0$\\
\begin{align*}
gcd(0,0) &= gcd(0,0-0)\\
&= gcd(0,0)
\end{align*}
\end{case}


\begin{case}
Assume at least one of $x,y$ is non zero\\
Suppose $d|x$ and $d|y$\\
We now must prove that $d|y-mx$\\
Since $(d|x \wedge y)(\exists k_0,k_1)$ such that $(x = d*k_0)(y = d*k_1)$ given $(k_0,k_1 \in Z)$
\begin{align*} 
\end{align*}
\end{case}
\end{proof}
\end{document} 